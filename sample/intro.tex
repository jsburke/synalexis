\section*{Introduction}
  Varmak is a minority language of Nogathirr originating from the Western Pampas. It is noted for some interesting morphological features such as stem alterarions, rich deixis distinctions, and multiple varieties of reduplication. The sound inventory is relatively simple as is the syllable structure. It has adopted a fair deal of its lexicon from the majority language, Swagathirr, although much of it has been either adapted for the current phonology or nativized into it over time such that they appear to be native terms.\par
  This work aims to provide the first complete bilingual lexicon on the language, which is a tremendous undertaking and was only possible given the warmth and hospitality of the Varmak speaking communities. To them we owe a deep debt of gratitude.\par
  The first major portion will enumerate English words with Varmak translations and usages as was fitting and available. The second major section is the reverse: a listing of terms in Varmak with translations into English with example usages as possible. The hope is to give not only a thorough account of words and translations, but also how these words interact to create the fullness of the Varmak language; we hope that it will do any readers or researchers the justice they need.\par
\vertspace

\section*{Phonology}
  Varmak features a fairly simple sound landscape with five pure vowels and a small handful of consonants. Not much detail of the sounds will be given here since that would be more fitting for a full language description. This section is presented only to make sense of the Latinization Scheme presented later and used throughout the lexicon.
\vertspace

  \subsection*{Vowels}
  \begin{tabular}{|l|l|l|l|}
    \hline
            & Front & Central & Back \\ \hline \hline
    High    & i     &         & u    \\
    Mid     & e     &         & o    \\
    Low     &       & a       &      \\ \hline
  \end{tabular}
  \vertspace
    The five vowel system is probably unsurprising. There are no tonal, length, or other distinctions that can be present. Within a larger linguistic sense, vowel alterations, such as lowering and "a-backing" which are triggered in various environments are fairly intriguing, but they will not be discussed here since they do not affect how terms will be logged in this work.
  \verstpace

  \subsection*{Consonants}
  \begin{tabular}{|l|l|l|l|l|}
    \hline
                  & Labial & Dental & Velar & Glottal      \\ \hline \hline
      Stop        & p      & t      & k     & \textipa{P}* \\ 
      Fricative   & f      & s      &       & h            \\ 
      Nasal       & m      & n      &       &              \\ 
      Rhotic      &        & r      &       &              \\ 
      Approximant &        &        & w     &              \\ \hline
  \end{tabular}
  \vertspace
  Varmak features 11 distinct consonants, a fairly simple yet full system. There are relatively few restrictions on them. Most dialects allow for word initial /r/, but some smaller ones do not, in which case they place an echo vowel before the initial /r/ found in most regions. Furthermore, the glottal stop /\textipa{P}/ only appears at the end of syllables, most frequently at the end of words.
  \vertspace

  \subsection*{Basic Phonotactics}
  Varmak has only (C)V(T) syllables allowed with a strong preference for CV. The T category of coda consonants allow for only /n, k, \textipa{P}/. Of these, /n/ is not uncommon word medially, but the other two are skewed towards the end of words.
  \vertspace

\section*{Latinization}
  The Latinization scheme matches the glyphs in the IPA with the exception of the glottal stop. Since it is limited to the end of syllables, and /t/ cannot appear there, it will be noted as a <t> in coda and word final positions.
\vertspace

% end of intro, start dictionary
\pagebreak
